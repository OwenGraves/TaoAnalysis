\documentclass[12pt]{article}
 
\usepackage[margin=1in]{geometry} 
\usepackage{amsmath,amsthm,amssymb}
 
\newcommand{\N}{\mathbb{N}}
\newcommand{\Q}{\mathbb{Q}}
\newcommand{\Z}{\mathbb{Z}}
 
\newenvironment{theorem}[2][Theorem]{\begin{trivlist}
\item[\hskip \labelsep {\bfseries #1}\hskip \labelsep {\bfseries #2}]}{\end{trivlist}}
\newenvironment{lemma}[2][Lemma]{\begin{trivlist}
\item[\hskip \labelsep {\bfseries #1}\hskip \labelsep {\bfseries #2}]}{\end{trivlist}}
\newenvironment{exercise}[2][Exercise]{\begin{trivlist}
\item[\hskip \labelsep {\bfseries #1}\hskip \labelsep {\bfseries #2}]}{\end{trivlist}}
\newenvironment{problem}[2][Problem]{\begin{trivlist}
\item[\hskip \labelsep {\bfseries #1}\hskip \labelsep {\bfseries #2}]}{\end{trivlist}}
\newenvironment{question}[2][Question]{\begin{trivlist}
\item[\hskip \labelsep {\bfseries #1}\hskip \labelsep {\bfseries #2}]}{\end{trivlist}}
\newenvironment{corollary}[2][Corollary]{\begin{trivlist}
\item[\hskip \labelsep {\bfseries #1}\hskip \labelsep {\bfseries #2}]}{\end{trivlist}}
\newenvironment{proposition}[2][Proposition]{\begin{trivlist}
\item[\hskip \labelsep {\bfseries #1}\hskip \labelsep {\bfseries #2}]}{\end{trivlist}}
 
\begin{document}

\title{Tao Analysis Chapter 3}
\author{Owen Graves}
\date{}
 
\maketitle
 
\begin{lemma}{3.1.13a}
	If $ a $ and $ b $ are objects, then $ \{a, b\} = \{a\} \cup \{b\} $.
\end{lemma}
\begin{proof}
	To prove the two sets are equal we must show that every element $ y $ of $ \{a, b\} $ is also an element of $ \{a\} \cup \{b\} $ and vice versa.
	From the definition of pair sets, $ y \in \{a, b\} $ if and only if $ y = a $ or $ y = b $.
	So first consider the case where $ y = a $, which implies that $ y \in \{a\} $ from the definition of a singleton set.
	From Axiom 3.4, we then have $ y \in \{a\} \cup \{b\} $.
	Now consider the case where $ y = b $, which implies that $ y \in \{b\} $ from the definition of a singleton set.
	Then again from Axiom 3.4, we have $ y \in \{a\} \cup \{b\} $.
	We now need to show that for every element $ y \in \{a\} \cup \{b\} $ we have $ y \in \{a, b\} $.
	From Axiom 3.4, we have $ y \in \{a\} \cup \{b\} $ if and only if $ y \in \{a\} $ or $ y \in \{b\} $.
	So first consider the case where $ y \in \{a\} $, which implies that $ y = a $ from the definition of a singleton set.
	Then we have $ y = a $ or $ y = b $ which is the definition of a pair set and so $ y \in \{a, b\} $.
	For the case where $ y \in \{b\} $, we also have $ y = b $.
	This entails that $ y = a $ or $ y = b $, which implies that $ y \in \{a, b\} $.
\end{proof}

\begin{lemma}{3.1.13b}
	The union operation is commutative, i.e, $ A \cup B = B \cup A $.
\end{lemma}
\begin{proof}
	From the definition of set equality, we must show that every element $ x $ of $ A \cup B $ is also an element of $ B \cup A $ and vice versa.
	Suppose that $ x \in A \cup B $.
	By Axiom 3.4, this means that either $ x \in A $ or $ x \in B $, which can be rewritten as $ x \in B $ or $ x \in A $.
	By Axiom 3.4 again, this is equivalent to $ x \in B \cup A $.
	We must now show for every element $ x \in B \cup A $ that $ x \in A \cup B $ also holds.
	So suppose $ x \in B \cup A $.
	Then we know that $ x \in B $ or $ x \in A $, and so $ x \in A \cup B $ as desired.
\end{proof}

\begin{lemma}{3.1.13c}
	The union operation is associative, i.e, $ (A \cup B) \cup C = A \cup (B \cup C) $.
\end{lemma}
\begin{proof}
	We must show that every element $ x $ of $ (A \cup B) \cup C $ is also an element of $ A \cup (B \cup C) $ and vice versa.
	So suppose that $ x $ is an element of $ (A \cup B) \cup C $.
	By Axiom 3.4 this means that either $ x \in A \cup B $ or $ x \in C $ is true.
	We now divide into cases.
	If $ x \in C $ then also $ x \in B \cup C $ by Axiom 3.4, and again using Axiom 3.4, $ x \in A \cup (B \cup C) $.
	Now suppose instead $ x \in A \cup B $, then by Axiom 3.4, either $ x \in A $ or $ x \in B $ and divide into cases again.
	If $ x \in A $, then $ x \in A \cup (B \cup C) $, by Axiom 3.4.
	If $ x \in B $, then $ x \in B \cup C $, by Axiom 3.4, and hence also by Axiom 3.4, $ x \in A \cup (B \cup C) $.
	Thus in all cases we have shown that every element of $ (A \cup B) \cup C $ is also an element of $ A \cup (B \cup C) $.
	A similar argument shows that every element of $ A \cup (B \cup C) $ lies in $ (A \cup B) \cup C $, and so $ (A \cup B) \cup C = A \cup (B \cup C) $ as desired.
\end{proof}

\begin{lemma}{3.1.13d}
	$ A \cup A = A \cup \emptyset = A $
\end{lemma}
\begin{proof}
	We first prove $ A \cup A = A $ by showing each set is a subset of the other.
	Let $ x $ be an element of $ A \cup A $.
	Then from Axiom 3.4 we have $ x \in A $ or $ x \in A $, and in both cases $ x \in A $ as desired.
	Now consider $ x \in A $, then we have $ x \in A $ or $ x \in A $.
	Then by Axiom 3.4, we have $ x \in A \cup A $ which completes this proof.
	
	We now prove that $ A \cup \emptyset = A $.
	Let $ x $ be an element of $ A \cup \emptyset $.
	From Axiom 3.4 this means that $ x \in A $ or $ x \in \emptyset $.
	If $ x \in A $, then clearly $ x \in A $.
	If $ x \in \emptyset $, then from Axiom 3.2 we have a falsehood and so $ x \in A $ is vacuously implied.
\end{proof}

\begin{proposition}{3.1.18a}
	Given sets $ A, B, C $, if $ A \subseteq B $ and $ B \subseteq C $ then $ A \subseteq C $.
\end{proposition}
\begin{proof}
	We need to show that if $ x \in A $ then also $ x \in C $.
	Suppose $ A \subseteq B $ and $ B \subseteq C $ and let $ x $ be an element in $ A $.
	From $ A \subseteq B $ we know that $ x \in B $.
	Since $ x \in B $ we know that $ x \in C $ from $ B \subseteq C $.
\end{proof}

\begin{proposition}{3.1.18b}
	Given sets $ A $ and $ B $, if $ A \subseteq B $ and $ B \subseteq A $ then $ A = B $.
\end{proposition}
\begin{proof}
	To prove $ A = B $ we must show that $ x \in A $ if and only if $ x \in B $.
	Suppose $ A \subseteq B $ and $ B \subseteq A $.
	From the definition of $ A \subseteq B $, we have $ x \in A $ implies $ x \in B $.
	Similarly, from the definition of $ B \subseteq A $, we have $ x \in B $ implies $ x \in A $.
	These two implications together allow the introduction of the biconditional $ x \in A $ if and only if $ x \in B $, thus completing the proof.
\end{proof}

\begin{proposition}{3.1.18c}
	Given sets $ A, B, C $, if $ A \subsetneq B $ and $ B \subsetneq C $ then $ A \subsetneq C $.
\end{proposition}
\begin{proof}
	We need to show that if $ x \in A $ then also $ x \in C $ and also that $ A \neq C $.
	Suppose $ A \subsetneq B $ and $ B \subsetneq C $ and let $ x $ be an element in $ A $.
	From $ A \subsetneq B $ we know that $ x \in B $.
	Since $ x \in B $ we know that $ x \in C $ from $ B \subsetneq C $.
	Thus we have shown that $ A \subseteq C $.
	To prove $ A \neq C $, we assume $ A = C $ and derive a contradiction.
	We have $ A \subseteq B $ from our assumption $ A \subsetneq B $.
	Since we assumed $ A = C $, we have $ B \subsetneq A $ and consequently $ B \subseteq A $.
	Since $ A \subseteq B $ and $ B \subseteq A $, we have $ A = B $ from Proposition 3.1.18b.
	However, this contradicts our previous assumption that $ A \neq B $, and so A must not equal C.
	Since we have shown that $ A \subseteq C $ and $ A \neq C $, we have proven that $ A \subsetneq C $.
\end{proof}

\begin{proposition}{3.1.28a}
	$ A \cup \emptyset = A $ and $ A \cap \emptyset = \emptyset $.
\end{proposition}
\begin{proof}
	The first equality follows from Lemma 3.1.13d.
	
	To prove the second equality we must show that every element of $ A \cap \emptyset $ is an element of $ \emptyset $ and vice versa.
	Consider $ x \in A \cap \emptyset $, then by the definition of set intersection, we also have $ x \in \emptyset $.
	Now consider $ x \in \emptyset $, but we also know that $ x \notin \emptyset $ from Axiom 3.2, which, from the principle of explosion, implies $ x \in A \cap \emptyset $.
\end{proof}

\begin{proposition}{3.1.28b}
	Consider a set $ X $ containing $ A $ as a subset.
	Then $ A \cup X = X $ and $ A \cap X = A $.
\end{proposition}
\begin{proof}
	We prove the first equality by showing $ A \cup X \subseteq X $ and $ A \cup X \supseteq X $.
	Consider $ x \in A \cup X $, then there are two cases.
	If $ x \in A $, then we also have $ x \in X $ since $ A \subseteq X $.
	In either case $ x \in X $ which shows that $ A \cup X \subseteq X $.
	Now consider $ x \in X $, then from disjunction introduction we have $ x \in A \cup X $.
	We have shown both subset inclusions and thus set equality.
	
	For the second equality we must show $ A \cap X \subseteq A $ and $ A \cap X \supseteq A $.
	Consider $ x \in A \cap X $, then from the definition of set intersection we have $ x \in A $.
	Now consider $ x \in A $, but then since $ A \subseteq X $, we have both $ x \in A $ and $ x \in X $ and thus $ x \in A \cap X $.
\end{proof}

\begin{proposition}{3.1.28c}
	$ A \cap A = A $ and $ A \cup A = A $.
\end{proposition}
\begin{proof}
	We prove the first equality by showing $ A \cap A \subseteq A $ and $ A \cap A \supseteq A $.
	Let $ x \in A \cap A $, then also $ x \in A $.
	Let $ x \in A $, then $ x \in A $ and $ x \in A $ so $ x \in A \cap A $.
	
	$ x \in A \cup A \iff x \in A \lor x \in A \iff x \in A $.
\end{proof}

\begin{proposition}{3.1.28d}
	$ A \cup B = B \cup A $ and $ A \cap B = B \cap A $
\end{proposition}
\begin{proof}
	
\end{proof}

\end{document}